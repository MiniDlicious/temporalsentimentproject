\chapter{Introduction}
\label{cha:introduction}


Opinons are often presented as the count of positive and negative views of the audience. This is common for elections, 
but also on social media platforms like Twitter or product reviews where it can be used for information retrieval \cite{twitter_mining}. 
However, for certain products or situations the opinion might change over time e.g. stock market or presidential elections. 
This is also more common for software products (e.g. applications, games or services) since these are continuously changed over time \cite{steam_reviews}.
As described in \cite{steam_reviews} this is an important view of the users and developers for either deeming if the game is a good investment or what 
issues to prioritize.




\section{Aim}
\label{sec:aim}

For this report the aim is to implement an automated approach for performing sentiment analysis on text data in a time-series approach. 
The outcome is a package with modular components that has the ability to digest text with a certain topic, classify each entry based on sentiment and then visulize with respect to time.
It is also of interest to use as simple methods as possible creating a modular framework for future improvement. 
Usability and understandability over complexity.


\section{Motivation}
\label{sec:motivation}

This project was chosen to utilize text mining for temporal insights of a topic, which can be important for decision making. This will also result in an usable application 
that can be used and improved in the future. 


\section{Delimitations}
\label{sec:delimitations}


This work is planned to take approximately 90 hours and audience is expected to have basic knowledge in Machine Learning and Text Mining. 
No ethical or societal aspects is included within the scope.