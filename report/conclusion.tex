\chapter{Conclusion}
\label{cha:conclusion}


The sentiment classification were very similar to the true labels and since some misclassifications are TN or FN, looking at a plot of both is very similar to the true plot. 
With a high accuracy of 83\% on validation data using Majority Voting the concept is working. 
The outcome is a modular package that can digest text, classify based on sentiment and visualize the results. 
Both code used and models were simple and would be easy to improve or change for the future. 
The application is a complete pipeline and fulfills the aim of the project.

For future work all components could be worked with separatly or together. 
With digestion and processing of text testing e.g. Term present, Term frequency–Inverse document frequency or Position of Speech tagging could improve the accuracy. 
The models have not been tuned, e.g. using grid search for better parametrization, which is suggested to improve performance without using more computational models. 
Then if there is computational resources other models such as Random Subspaces, Bagging or switching to deep learning models should improve performance. 
However, it is proved that simple models can give a good result for everyday applications.