\chapter{Discussion}
\label{cha:discussion}


This chapter analyse the results obtained and the method used, both described in respective chapter.


\section{Results}
\label{sec:discussion-results}


The results from training and testing the models are closely related to the results seen in related work, described in chapter \nameref{cha:theory}. 
The KNN model was not used in any of the referenced works, but was included since it is a common ML model and results in non-linear classification borders. 
It was unexpected to see that KNN performed so unbalanced, classifying negative reviews so well, but very poor on positive reviews. 
Also the potential of Random Subspaces (RS) is hinted, it is used in a poorly parameterized way, but the generalization i.e. performance on test data is very good considering. 
However, since it uses SVM as base model it should still be close to the Linear SVM model that is a more basic model than the SVM used in RS. 


The majority voting incresed performance as expected from referenced papers in a similar way. 
It could be argued that this increases complexity over just using one model, but it also balanced the FP and FN classifications. 
Possibly using majority voting the classifications will be more stable over different datasets, this would require more investigation and could be future work. 
Using low complexity models made the prediction time fast even when using the voting system.


\subsection{Validation}
\label{sec:discussion-validation}


Since classifications of the validation dataset was good with over 80\% accuracy, the dataset used for training was representative enough. 
The games, Wolcen and GTA V, are both similar and very different being in different sub-genres. 
However, both are games and reviews resemble one another but genre specific words are probably not included in training. 
The result could possibly be improved with a broader set of games used for training. 


The models performed otherwise as expected with majority voting having the highest accuracy, just as described in related research. 
Analyzing the plots created had very similar result, since FP and FN is balancing cancelling each other when looking at the figures. 
It also looks like the misclassifications are distributed along the time axis, not causing any large difference in the overall shapes. 


As for the most common positive and negative words they are quite expected. 
Writing \emph{good}, \emph{great} or \emph{fun} in a positive review or \emph{bugs} in a negative review is probably common. 
It looks like \emph{just} is very common in negative reviews, which might not be expected. 
This is probably due to writing \emph{...just like *another game*..} or \emph{...just too many bugs...}, this analysis could also be future work.


\section{Method}
\label{sec:discussion-method}

Analyse your results and discuss the possibilities and limitations of
your technical approach. Compare your study to related work.

This is where the applied method is discussed and criticized.
Taking a self-critical stance to the method used is an
important part of the scientific approach.

A study is rarely perfect. There are almost always things one
could have done differently if the study could be repeated or
with extra resources. Go through the most important
limitations with your method and discuss potential
consequences for the results. Connect back to the method
theory presented in the theory chapter. Refer explicitly to
relevant sources.

The discussion shall also demonstrate an awareness of methodological
concepts such as replicability, reliability, and validity. The concept
of replicability has already been discussed in the Method chapter
(\ref{cha:method}). Reliability is a term for whether one can expect
to get the same results if a study is repeated with the same method. A
study with a high degree of reliability has a large probability of
leading to similar results if repeated. The concept of validity is,
somewhat simplified, concerned with whether a performed measurement
actually measures what one thinks is being measured. A study with a
high degree of validity thus has a high level of credibility. A
discussion of these concepts must be transferred to the actual context
of the study.

The method discussion shall also contain a paragraph of
source criticism. This is where the authors’ point of view on
the use and selection of sources is described.

In certain contexts it may be the case that the most relevant
information for the study is not to be found in scientific
literature but rather with individual software developers and
open source projects. It must then be clearly stated that
efforts have been made to gain access to this information,
e.g. by direct communication with developers and/or through
discussion forums, etc. Efforts must also be made to indicate
the lack of relevant research literature. The precise manner
of such investigations must be clearly specified in a method
section. The paragraph on source criticism must critically
discuss these approaches.

Usually however, there are always relevant related research.
If not about the actual research questions, there is certainly
important information about the domain under study.
