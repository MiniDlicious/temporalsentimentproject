\chapter{Method}
\label{cha:method}



Explain how you carried out your study. Aim to be detailed enough for
others to reproduce your results.


In this chapter, the method is described in a way which shows how the
work was actually carried out. The description must be precise and
well thought through. Consider the scientific term
replicability. Replicability means that someone reading a scientific
report should be able to follow the method description and then carry
out the same study and check whether the results obtained are
similar. Achieving replicability is not always relevant, but precision
and clarity is.

Sometimes the work is separated into different parts, e.g.  pre-study,
implementation and evaluation. In such cases it is recommended that
the method chapter is structured accordingly with suitable named
sub-headings.


delat i moduler: extraction, classification and evaluation

\section{Pre-study}
\label{sec:prestudy}

similar research ..

steam reviews ..

Text extraction ..

Sentiment classification ..

evaluation ..


\section{Implementation}
\label{sec:implementation}


\section{Evaluation}
\label{sec:evaluation}

For evaluation the similar Steam Reviews will be used for a game the author has knowledge about. 
The author will predict the behaviour of the reviews based on knowledge about the game and by reading update notes from the developers to see what issues have been present in the game. 
Then in this section there will be a prediction made before investigating the reviews and looking at the graphs presented by Steam. 
This will later be used as a benchmark for the analysis package under \textcolor{red}{section YY}.


The game chosen is a PC game named \emph{Wolcen: Lords of Mayhem}, which started as a kickstarter with the name \emph{Umbra}. \cite{kickstarter} 
The game is available at Steam \cite{wolcen} and was released on Febuary 13 2020 after a 4 year early access for a limited audience. 
The audience had very high hopes for the game from the very successful kickstarter campaign and glimpses of the game during development. 
However, the release was not as smoothed as anyone could have hoped for with server issues and unstable gameplay. 
At the same time it was a somewhat new take on the genre and had many positive elements that could evolve into the game that was expected. 


The development team then released small patches and hotfixes to the game without any new content for almost a year. 
The 3 December 2020 a large new content update was released that have been very sought for in the community. 
This were another implementation of what was promised during the kickstarter campaign along with rebalancing and restarting the game and economy that had been affected by the early bugs at release. During the last month more hotfixes and small pathces has been released without any new content.

At the time of writing the game has 54 000 reviews which will be used for evaluating the work performed in this report. 
With the information about the game, over 100 hours of gameplay and looking over the patch notes and updates certain patterns are expected. 
There has been lots of game breaking issues around release of the game, suggesting that there should be a fair amount of negative reviews. 
Also looking at where in the development process according to the kickstarter campaign, these are not yet fulfilled at release meaning that content is missing that has been promised. 

During a game release there will always be most active players just after release and later if there is large content releases. 
This is where peaks in the reviews are expected, both negative and positive. 
There should be several magnitudes of reviews more around release date than around the large content update in December. 
It is also expected that there are a very low number of reviews in between these releases. 
Since the content update still not completed the kickstarter roadmap, there is probably a high number of negative reviews at the time for this update.
